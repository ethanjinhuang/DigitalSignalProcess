\documentclass[cn,hazy,blue,14pt,screen,normal]{elegantnote}
\title{数字信号处理(Digital Signal Process)笔记}
\author{Jin Huang}
\usepackage{array}
\usepackage{ctex}

\begin{document}
\maketitle
\newpage
\section{线性时不变(LTI)系统}
\subsection{线型时不变系统的定义及响应}
线性时不变系统的特性:可加权、可叠加、可时移
\begin{equation}
    y(n) = ay_1(n)+by_2(n-n_0) \label{1}
\end{equation}
\textbf{线性时不变系统的定义:}
\begin{itemize}
    \item 同时满足线型和时不变性的系统。
\end{itemize}
\textbf{系统的输出响应分类:}
\begin{itemize}
    \item[1. ]  零状态相应$y_{zs}(n)$:仅由输入序列在观察时刻之后产生的相应。(线性时不变的)
    \item[2. ]  零输入相应$y_{zi}(n)$:仅由初始状态引起的相应。(非线性的)
    \item[3. ]  全响应$y(n) = y_{zs}(n)+y_{zi}(n)$。(非线性的)
\end{itemize}
\begin{equation}
    y(n) = x(n) + b \label{2}
\end{equation}
公式\ref{2}中,$x(n)$表示输入,$b$表示初始状态。

\subsection{单位脉冲响应}
\textbf{定义:}当系统输入为$\delta(n)$时,系统的零状态相应称为\emph{单位脉冲响应},记为$h(n)$。

\qquad 对于$y(n) = T[x(n)]$ \qquad 有$h(n) = T[\delta(n)]$

换言之,对于一个线性时不变系统,输入一个单位脉冲序列$\delta(n)$,输出一个单位脉冲相应$h(n)\ref{3a}$;
如果输入序列移位,则输出响应也对应移位\ref{3b};如果输入加权,则输出相应也对应加权\ref{3c}。
\begin{subequations}
\begin{align}
    \delta(n) = h(n) \label{3a} \\
    \delta(n-1) = h(n-1) \label{3b} \\
    a \delta(n-2) = ah(n-1) \label{3c}
\end{align}
\end{subequations}

\subsection{线性时不变系统的输入输出运算}
在\textbf{线型时不变系统}中,有一乘法器,乘法器的两个输出分别为$x(n)$和$p(n) = \sum\limits^{+\infty}_{m=-\infty}{\delta(n-m)}$。依次考虑各项
$$x(n)p(n) = y(n)$$
$$p(n) = \sum\limits^{+\infty}_{m=-\infty}{\delta(n-m)}$$
\begin{table}[htbp]
    \centering
    \caption{系统输入输出参照}
    \begin{tabular}{lll}
        \hline
        m   &$p(n)$     &输出\\
        \hline
        m=0 &$\delta(n)$    & $x(0)h(n)$\\
        m=1 &$\delta(n-1)$  & $x(1)h(n-1)$\\
        m=2 &$\delta(n-2)$  & $x(2)h(n-2)$\\
        $\dots$ & $\dots$ &$\dots$ \\
        \hline
            & $\sum\limits^{+\infty}_{m=-\infty}{\delta(n-m)}$ & $\sum\limits^{+\infty}_{m=-\infty}{x(m)h(n-m)}$\\
        \hline
    \end{tabular}
\end{table}

\begin{equation}
    y(n) = x(n)*h(n) = \sum\limits^{+\infty}_{m = -\infty}{x(m)h(n-m)} \label{4}
\end{equation}

由公式\ref{4}可看出,序列输入系统的运算是卷积和。

\begin{itemize}
    \item [1)] 线性时不变系统的输入输出运算关系
            $$y(n) = x(n)*h(n)$$
            用卷积和运算描述系统输入输出关系的注意事项:
            \begin{itemize}
                \item [(1)]系统必须是线性时不变系统
                \item [(2)]所求输出为系统的零状态相应
            \end{itemize}
    \item [2)] 单位脉冲响应可以描述线型时不变系统的零状态响应特征
\end{itemize}

\end{document}